\documentclass[a4paper]{article}
\usepackage{ctex}
%\usepackage{svg}
\usepackage{amssymb,amsmath}
\usepackage{geometry}
\usepackage{graphicx}

\title{魔法定级}
\date{\today}
\author{RyanBern}

\begin{document}
\maketitle
\textbf{说明}
\begin{itemize}
\item 以下的每个魔法题目前面都有标注难度,难度的含义是相对于同类的魔法题目,并不是绝对的难度。
题目中涉及到的几何图形均省略,因为这些图形都很简单,描述起来不会产生歧义。
\item 定级时需要根据自己目前正在攻读的学位进行选择,但可以选择自己攻读学位以上的等级。本科/研究生
都被归入“大学”一级。
\item 不同级别需要的题目数量不同。初中为3道;高中为6道;大学为9道。题目中至多选择两道处于自己
等级以下的题目(例如如果要定成“大学”等级,那么至多只能选择两道初中或者高中的题目)。
\item 大学魔法的前两大题,每小问算半个题目。
\item 定级时,考官会根据提交答案的次数,选择问题的难度以及解法是否醋来对定级进行调整。
\end{itemize}

\section{初中魔法}
\begin{enumerate}
\item (easy)有54张纸牌,编号为1到54,初始按照编号从小到大的顺序依次放好。现在扔掉第1张,然后将第2张放到最下面,
扔掉第3张,将第4张放到最下面,如此往复,直到只剩下一张牌为止。问最后剩下编号为多少的牌?
\item (easy)从正方体 8 个顶点中随机取三个点,则构成等腰三角形的概率为?
\item (medium)对平面内的$\triangle ABC$,存在同平面的点$P$,使得$\triangle PAB,\triangle PAC,\triangle PBC$
面积相等。这样的点$P$有多少个?
\item (medium)求证:$1+\frac{1}{2^2}+\frac{1}{3^2}+\cdots + \frac{1}{n^2}<\frac{7}{4}$。
\item (hard)设$\triangle ABC$是等腰三角形,$AB=AC,\angle A=80^\circ$。点$O$为$\triangle ABC$内部
一点,且$\angle OBC=10^\circ,\angle OCA=20^\circ$。求$\angle BAO$的度数。
\end{enumerate}

\section{高中魔法}
\begin{enumerate}
\item (easy)设$a,b,c$是实数,则$a^3+b^3+c^3=3abc$的条件为?
\item (easy)将四个半径为1的小球堆起来,两两相切,那么这个几何体的外切四面体的边长为?
\item (easy)求证:$xy=1$在某种直角坐标替换下能够变成$u^2/a^2-v^2/b^2=1$的形式。
\item (easy)三角形三个顶点对应的复数为$z_1,z_2,z_3$,并且有$\frac{z_2-z_1}{z_3-z_1}=1+2i$,则
其面积和最长边的平方之比是?
\item (medium)求值:$\cos\frac{2}{7}\pi+\cos\frac{4}{7}\pi+\cos\frac{6}{7}\pi$
\item (medium)设椭圆$E:\frac{x^2}{a^2}+\frac{y^2}{b^2}=1,a>b>0$,点$P(x_0,y_0)$在椭圆
上。过点$P$分别引出两条斜率为$k_1,k_2$的直线,满足$k_1+k_2=0$。两条直线分别交椭圆于点$A$和$B$。
求证:$AB$的斜率是定值。
\item (medium)设定义在实数上的函数$f$满足$f(f(x))=x$恒成立,并且$f$是单调递增。求证:满足条件的$f$存在且唯一。
\item (hard)设数列$\{a_n\}$满足递推式$a_{n+1}=a_n^2-2,a_0=a$。求$a_n$的表达式。
\item (hard)空间点集$A_n=\{(x,y,z)\in\mathbb{R}^3:\; 3|x|^n+|8x|^n+|z|^n \leqslant 1\}$,令
$A=\bigcup_{n=1}^{\infty}A_n$,则$A$表示的几何体的体积是?
\end{enumerate}
\section{大学魔法}
\begin{enumerate}
\item (easy)下列说法是否正确?请证明或者证伪之。
	\begin{enumerate}
	\item 复值函数$\sin(z)$是有界的。
	\item 设函数$f(x)$在$\mathbb{R}$上可导,并且有$\lim_{x\rightarrow +\infty}f(x)=0$,
	则有$\lim_{x\rightarrow +\infty}f'(x)=0$。
	\item 可导函数$f(x)$在$\mathbb{R}$上为凸函数当且仅当$f'(x)$是单调递增的。
	\item 一元函数列$f_n$一致收敛与$f$,并且$f_n$和$f$均是无穷次可导,则$f_n'$也一致收敛于$f'$。
	\item 在$n\times n$矩阵构成的线性空间中,可逆矩阵所组成的集合是道路连通的。
	\end{enumerate}
\item (easy)试构造出符合条件的函数或集合。
	\begin{enumerate}
	\item $f(x,y)$在$\mathbb{R}^2$上的两个偏导数处处存在,但$f$在$(0,0)$和$(0,1)$处无界,
	在其他点两个偏导数连续。
	\item 可微多元函数$f$在某点$x_0$处,对于任意向量$v$,$t=0$均为$g(t)=f(x_0+tv)$的极小值点。
	但$x=x_0$不是$f$的极小值点。
	\item 请构造$\mathbb{R}^2$上的子集$E$,使其满足存在$E$边界上的点$A$,使得对于任意$E$的内点$B$,
	$A$与$B$不是道路连通的。
	\end{enumerate}

\item (easy)设定义在$(0,1)$的函数$f$满足$\sup_{x,y}\frac{|f(x)-f(y)|}{|x-y|^\alpha}<M$,其中
$\alpha>1$,$M$是一个有限的常数。求证$f$在$(0,1)$上是常值。
\item (easy)求证$\|x+y\|_2=\|x\|_2+\|y\|_2$当且仅当$x,y$线性相关且$x^Ty\geqslant 0$。
\item (easy)已知矩阵$A$正定,矩阵$B$半正定,求证$A,B$可以在同一个合同变换下对角化。
\item (medium)设$n$元函数
\[
f(x)=\max x_k + \frac{1}{2}\sum_{k=1}^{n}(x_k-a_k)^2
\]
求$f(x)$的一个最小值点。
\item (medium)若$\|A\|<1$,并且$\|I\|=1$,其中$I$为单位阵,$\|\cdot\|$为任意矩阵范数。求证$I-A$可逆并且
\begin{equation}
\|(I-A)^{-1}\|\leqslant \frac{1}{1-\|A\|}
\end{equation}
\item (medium)设二元函数$f(x,y)$是凸函数,$C$是凸集。令$g(x)=\inf_{y\in C}f(x,y)$,求证:
$g(x)$为凸函数。
\item (medium)设矩阵$A$为循环矩阵,即
\begin{equation}
A=\left(\begin{array}{cccc}
a_0 & a_1 & \cdots & a_{N-1} \\
a_1 & a_2 & \cdots & a_0     \\
\vdots & \vdots & \ddots & \vdots \\
a_{N-1} & a_0 & \cdots & a_{N-2} \\
\end{array}
\right)
\end{equation}
试给出一个快速求解方程$Ax=b$的算法。
\item (hard)设$Y_n$是一列独立同分布的随机变量,并且有$\mathbb{E}|Y_1|>0$。定义$S_n=Y_1+\cdots+Y_n$。
设随机变量$T=\inf\{n:\; |S_n|>1 \}$(即$T$定义为第一个$n$使得$S_n$不在区间$[-1,1]$内)并规定$\inf\phi=\infty$。
求证,存在正数$c$和$0<r<1$,使得$\mathbb{P}(T>n)\leqslant cr^n$对于任意的$n$恒成立。
\end{enumerate}
\end{document}