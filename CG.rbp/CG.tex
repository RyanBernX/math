\documentclass[a4paper]{article}
\usepackage{amsmath, amssymb}

\title{Problems for Conjugate Gradient Method in Solving Linear Systems}
\author{RyanBern}
\date{\today}

\begin{document}
\maketitle
\begin{enumerate}
\item Let $\{d_0, d_1, \ldots, d_{k-1}\}$ be a set of vectors which satisfy
$d_i^TAd_j=\delta_{ij}$, where $A$ is positive definite and $\delta_{ij}$ is
the Kronecker delta. Suppose that we have searched the minimum value along
each $d_i$ sequentially(therefore we have a sequence of points $x_0, x_1, \cdots, x_{k-1}$).
Let $g_j=Ax_j-b$ be the gradient of $f$ at $x=x_j$, prove that $g_k^Td_j=0,~~j=0,1,\cdots,k-1$.

\emph{Hint: if we do linear search along $x_{k-1}+t_{k-1}d_{k-1}$, then at the optimal $t_{k-1}$, we have
$g_{k}^Td_{k-1}=0$}.
\item Let $\{d_0, d_1, \ldots, d_{k-1}\}$ be a set of vectors which satisfy
$d_i^TAd_j=0$ for any $0\leqslant i<j\leqslant k-1$. $g_j=Ax_j-b$ is the gradient of $f$ at $x=x_j$. Furthermore, suppose
$d_0=-g_0$. Let $d_k=-g_k+\sum_{j=0}^{k-1}a_jd_j$ be the searching direction at step $k$
which satisfies $d_k^TAd_j=0$ for all $j=0,1,\ldots,k-1$. Prove that $a_j=0,~~j=0,1,\ldots,k-2$.

\item Compute the optimal $t_k$ and $a_{k-1}$ at step $k$. Your results should have
a simple form which only includes matrix-vector production and vector-vector production.

\item Implement the conjugate gradient algorithm for optimizing the quadratic function and
test your program with the following examples.
\begin{enumerate}
\item $A\in \mathbb{R}^{n\times n}$, where $n=60000$, and has the form
\[
A=\left(\begin{array}{ccccc}
10 & 1 &&&\\
1 & 10 & 1 &&\\
&1&\ddots&\ddots &\\
&&\ddots&\ddots& 1\\
&&& 1 & 10\\
\end{array}
\right)
\]
$b$ and $x_0$ can be chosen randomly.
\item $A=(a_{ij})\in \mathbb{R}^{n\times n}$, where $n=40$ and $a_{ij}=\frac{1}{i+j-1}$.
$b=(b_i)$, where $b_i=\sum_{j=1}^{n}a_{ij}$. $x_0$ is chosen randomly. Obviously, the
solution of $Ax=b$ is $x=(1,1,\ldots,1)^T$. Apply the Gaussian elimination with pivoting
to solve the linear system again, and explain what you've observed.
\end{enumerate}

\end{enumerate}
\end{document}