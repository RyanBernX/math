\documentclass[a4paper]{article}
\usepackage{amsmath, amssymb}

\title{Problems for QR Factorization}
\author{RyanBern}
\date{\today}
\begin{document}
\maketitle
\begin{enumerate}
\item \emph{(Easy)}Consider the normal equation
\[
X^TX\beta = X^TY
\]
where $X\in\mathbb{R}^{m\times n},m \geqslant n$ and $Y\in\mathbb{R}^{m}$.
Prove that the solution set of the equation is non-empty. \emph{Hint: you may
have to show that the ranks of the coefficient matrix and the augmented matrix
are equal.}

\item \emph{(Medium)}Review the Householder QR factorization and answer the following questions.
\begin{enumerate}
\item Let $X\in\mathbb{R}^{m\times n}, m \geqslant n$, then what's the complexity
of Householder QR in terms of arithmetic operations? Only the highest degree
terms are needed.
\item Let $X\in\mathbb{R}^{2n\times n}$, and be composed of two upper-triangular
matrices, i.e.
\[
X=\left(\begin{array}{c}
R_1 \\
R_2 \\
\end{array}
\right)
\]
where $R_1$ and $R_2$ are upper-triangular. Write down an efficient algorithm to
perform Householder QR factorization on $X$. What's the complexity of your
algorithm? \emph{Hint: you should make use of the special structure of $X$.}
\end{enumerate}

\item \emph{(Hard, Programming)} Practical Householder QR Factorization. In most
numerical algebra softwares such as \textsc{lapack}, QR factorization is implemented
in a slightly different way. Recall the definition of Householder transformation,
$H=I-2vv^T$, where $v\in\mathbb{R}^{m}$ is a vector and has unit Euclidean norm.
Most softwares use a different definition of $H$ instead of $H=I-2vv^T$. They prefer
$H=I-\tau uu^T$ where $u$ is a vector in $\mathbb{R}^{m}$ and satisfies $u_1=1$,
$\tau$ is a scalar called Householder multiplier. In practical QR, the $Q$ factor
is stored implicitly, by overwriting the strict lower trapezoidal part of $X$ with
the scaled Householder vector $u_i$. And the upper triangular part is overwritten
by the $R$ factor exactly. Note that the first element of $u_i$ is 1 so
there's no need for storing it. Also, an extra array of $\tau$ is needed to store
those Householder multipliers. For example, if $X$ has 5 rows and 4 columns, after
QR factorization, $X$ should be overwritten by
\[
X=\left(\begin{array}{cccc}
r_{11} & r_{12} & r_{13} & r_{14} \\
u_{21} & r_{22} & r_{23} & r_{24} \\
u_{31} & u_{32} & r_{33} & r_{34} \\
u_{41} & u_{42} & u_{43} & r_{44} \\
u_{51} & u_{52} & u_{53} & u_{54} \\
\end{array}
\right)
\]
Also, the $\tau$ array should be $(\tau_1, \tau_2, \tau_3, \tau_4)$.
\begin{enumerate}
\item Derive the expression of $\tau$ and $u$ such that $Hx=\alpha e_1$, where
$x$ is given and $e_1=(1, 0, \ldots, 0)^T$.
\item Implement the practical Householder QR.
\item Suppose we need the $Q$ factor explicitly, show how to obtain the explicit
$Q$ factor from the implicitly stored Householder vectors. Implement your algorithm.
\end{enumerate}
\item \emph{(Medium, Programming)}Consider the Hilbert matrix of order 20. Orthogonalize
the matrix using Gram-Schmidt process and QR factorization respectively and check
your results. Are these computed matrices numerically orthogonal?
\end{enumerate}
\end{document}